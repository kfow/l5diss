using a data-driven approach -- oft

% Sensitivity review is a task carried out by government departments in the process of preparing documents for release to the general public. 
% If a document is deemed to contain sensitive information then it may be withheld or redacted (this can also occur for other reasons)~\footnote{\url{https://ico.org.uk/for-organisations/guide-to-freedom-of-information/refusing-a-request/}}.
% % Here you appear to say the same thing twice.
% There are many reasons a document maybe be deemed sensitive\footnote{\url{http://www.legislation.gov.uk/ukpga/2000/36/part/II}}, however some examples are documents containing information which could negatively affect international relations or documents containing information which could put members of the armed or security forces at risk.

% Part of this process involves cross checking document content with public domain information,  
% Increasingly the documents which require review are digital and these documents are being generated at an ever increasing rate due to new technologies~. 
% \remove{The National Archives also must decide which documents are worth keeping which in itself poses an interesting problem~\cite{moss2012have}, demonstrating further the scale of the these document collections.}

\begin{table}
\begin{tabular}{l||c||p{2cm}}
\emph{Operating System} & \emph{Version} & \emph{Verdict} \\ \hline \hline
Ubuntu & 12.04 & Everyone's favourite Linux, unless you grew up with
RedHat \\ \hline
Slackware & xxx & Pseudo-hacker's Linux, how often do you recompile
your kernel? \\ \hline
Mac OS & 10.7 & For people with more money than sense \\ \hline
\end{tabular}
\caption{\label{tab-eg}Single column table of figures}
\end{table}

\begin{figure*}
\begin{center}
\includegraphics[scale=0.3]{alice.pdf}
\end{center}
\caption{\label{fig-eg}An example figure stretching over two columns}
\end{figure*}

\begin{enumerate}
\item General description of the problem, motivation, relevance
\item Background information, possibly including a literature survey
\item Description of approach taken to solve the problem, including
  high-level design and lower-level implementation details as appropriate
\item Evaluation, qualitative or quantitative as appropriate
\item Conclusion, including scope for future work
\end{enumerate}

\begin{center}
\begin{table}[h]
\centering
\resizebox{\columnwidth}{!}{%
\begin{tabular}{|c|c|c|c|c|c|}
\hline
% These feature lists do not contain other as a feature!
% Purely a test on the reranking capabilities of named entities as features
Feature Set	 			& MAP 			  & MRR  			& P@5 	      	  \\ \hline
All w/ TF   			& 0.1376 		  & 0.5828    		& 0.3685          \\ \hline
Persons TF				& 0.1432    	  & 0.6014    		& 0.3901          \\ \hline
Organisations TF		& \textbf{0.1559} & 0.5871 			& 0.3784	      \\ \hline
Locations TF 			& 0.1550 		  & 0.6267    		& \textbf{0.4289} \\ \hline
Persons and Orgs TF 	& 0.1351 		  & 0.5568    		& 0.3475	      \\ \hline
Persons and Locs TF 	& 0.1510 		  & \textbf{0.6280} & 0.4222	      \\ \hline
Locations and Orgs TF 	& 0.1497 		  & 0.5913   	 	& 0.3851	      \\ \hline
\end{tabular}%
}
\caption{Old results before we worked out sampling.}
\label{ne_results}
\end{table}
\end{center}


\begin{center}
\begin{table}[h]
\centering
\resizebox{\columnwidth}{!}{%
\begin{tabular}{|c|c|c|c|c|c|}
\hline
Retrieval Model	 		& MAP 			  	& MRR  			& P@5 	      	  \\ \hline
DPH	  					& 0.1727 		  	& 0.6406    	& 0.4347          \\ \hline
BM25					& 0.1824    	  	& 0.6304    	& 0.4277          \\ \hline
TF-IDF					& 0.1787 			& 0.6310 		& 0.4257	      \\ \hline
PL2 					& 0.1798 		  	& 0.6271    	& 0.4195 		  \\ \hline
TF 						& 0.0716 		  	& 0.3989    	& 0.2245	      \\ \hline
\end{tabular}%
}
\caption{Basic Textual Query Results Using Different Models}
\label{plain_results}
\end{table}
\end{center}

\begin{center}
\begin{table}[h]
\centering
\resizebox{\columnwidth}{!}{%
\begin{tabular}{|c|c|c|c|c|c|}
\hline
Retrieval Model	 		& MAP 			  	& MRR  			& P@5 	      	  \\ \hline
DPH	  					& 0.1737 		  	& 0.6410    	& 0.4335          \\ \hline
DPH w/ First 100		& 0.1061    	  	& 0.5456    	& 0.3286          \\ \hline
\end{tabular}%
}
\caption{Index with no temporal resolution or named entity recognition, using full document text as queries.}
\label{plain_index_plain_results}
\end{table}
\end{center}